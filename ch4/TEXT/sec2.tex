%%%%%
%%Title: HiPi+Bus V0.2 Chapter 4
%%Creator: Ando Ki
%%CreationDate: April 1992
%%FileName: sec2
%%RelatedFile: ch4
%%%%%
\section{인터럽트 버스의 신호선}
%
%\documentstyle[a4]{hbook}
%\begin{document}
%
\begin{table}[htbp]
\caption{인터럽트 버스의 신호들}\label{table:ib-signal}
   \begin{center}
   \begin{tabular}{|l|l|l|} \hline
	Mnemonic & Size & Name \\
\hline \hline
	IBSYNC*                & 1 & Interrupt Bus Sync. \\
	IBD{\tt <}7..0{\tt >}* & 8 & Interrupt Bus Data \\
	IBDP*                  & 1 & Interrupt Bus Data Parity \\ \hline
   \end{tabular}
   \end{center}
\end{table}
%
%\end{document}

%
%
\subsection{Interrupt Bus Sync : IBSYNC*}
본 버스의 모든 동작은 시분할 방식으로 진행되기 때문에 필요한 신호이다.
이 신호는 인터럽트를 전송하고 있는 모듈이 구동하는 신호로써, 
이제 인터럽트 전송이 끝나므로 다음 인터럽트 전송을 원하는 모듈은 중재를 준비하라는 의미를 갖는 신호이다.
인터럽트를 전송하고 있는 모듈은 이 신호를 계속 ``참 (낮은 전압)''으로
구동하며, 인터럽트 전송 마지막 주기에서 이 신호를 ``거짓 (높은 전압)''으로 한다.
인터럽트를 전송하고자 기다리고 있는 모듈들은 이 신호를 관찰하여 ``거짓''이 되면
중재를 수행하게 된다. 중재를 수행하는 모듈도 이 신호를 ``참''으로 구동한다.
중재 동작 후 버스 사용허가를 받은 모듈은 자신이 인터럽트 전송을 마칠 때까지 ``참''으로 구동하게 되고,
나머지 모듈은 구동을 멈추고 전송이 끝나서 이 신호가 ``거짓''이 될 때까지 기다리게 된다.
따라서 인터럽트를 전송하고자 하는 모듈이 없을 때는 이 신호의 상태는 ``거짓''이 된다.
%
\subsection{Interrupt Bus Data : IBD{\tt <}7..0{\tt >}*}
이 신호는 크게 두가지 목적을 갖는데, 첫째는 인터럽트에 관한 데이터 
전송 통로이고, 둘째는 버스 중재를 위한 제어 신호이다.
따라서 이 신호는 인터럽트 전송 주기에 따라 다른 방법으로 동작한다.
이 신호는 인터럽트 버스 상의 모든 모듈이 입출력으로 사용할 수 있고,
동시에 두개 이상의 모듈이 구동할 수 있으며 중재 기능을 구현할 수 있도록
wired-OR 신호선을 사용한다. \\
데이터 전송 통로로 이용될 경우는 동기형 프로토콜의 일반적인 데이터 전송 방법을 사용한다.
인터럽트 전송 주기에 따라 전송 방향이 결정되고, 데이터를 전송하는 모듈은
한개가 되며, 한개 이상의 모듈로 입력이 될 수 있다.
인터럽트의 종류 (Class), 요청기의 주소 (Source Address), 
벡터 (Vector), 응답 (Acknowledge)등이 전송된다. \\
인터럽트 버스의 중재 동작 중에는 요청기들 사이의 중재와 처리기 사이의 중재가 있다.
요청기 사이의 중재는 같은 주기에 두개 이상의 요청기가 인터럽트를 전송하고자 할 경우 
인터럽트 전송의 시작 전에 이루어지며,
처리기 사이의 중재는 인터럽트가 여러개의 처리기로 보내지는 중재 인터럽트 주기 중에 
인터럽트 처리를 맡을 한 처리기를 선정하는 과정에 수행된다.
중재 동작에서 이 신호들로 출력되는 정보는 인터럽트의 종류와 슬롯 번호등이 되며,
각 신호는 중재 버스와 같이 각 비트 별로 우선 순위 비교를 위해 사용된다.
%
\subsection{Interrupt Bus Data Parity : IBDP*}
인터럽트 버스의 신뢰도 향상을 위하여 필요한 신호로써
IBD{\tt <}7..0{\tt >}*의 패리티로 동작된다.
중재동작 때는 패리티 사용하지 않고, 데이터 전송동작 때는 해당 데이터의 패리티로 동작한다.
%%
%%
