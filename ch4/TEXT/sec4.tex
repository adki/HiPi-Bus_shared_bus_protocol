%%%%%
%%Title: HiPi+Bus V0.2 Chapter 4 Section 4
%%Creator: Ando Ki
%%CreationDate: April 1992
%%FileName: sec4
%%RelatedFile: ch4
%%%%%
\section{벡터의 종류}\label{section:vector-type}
인터럽트 벡터는 전송된 인터럽트의 중심이 되는 정보로써 8 비트와 16 비트 크기가 있다.
인터럽트 벡터가 담고 있는 의미는 인터럽트 발생 원인과 요청하는 동작이 무엇인지를
나타내게 된다. 따라서 각 벡터는 전 시스템에서 유일한 의미를 갖게 된다.
그와 같은 벡터는 그들이 갖고 있는 특성에 따라
4 가지로 분류되는데, 전송된 정보 중에 2 비트의 Type 필드에 의해 지정된다.
%\documentstyle[a4]{hbook}
%\begin{document}
%
\begin{table}[htbp]
\caption{벡터의 종류}\label{table:vec-type}
   \begin{center}
   \begin{tabular}{|l|l|l|l|l|l|l|} \hline
	vector & name   & vector & description & priority & mask & class \\
	type   &        & size   &              &          &      & \\
\hline \hline
	3      & E-type & 8bit   & emergency int. & high & non-maskable & DI \\
\hline
	2      & I-type & 8bit   & immediate funciton & & no pending & DI \\
	       &        &        &                    & & maskable &  \\
\hline
	1      & N-type & 16bit  & notify int. & & no pending & DI or AI \\
	       &        &        &             & & maskable & \\
\hline
	0      & Q-type & 16bit  & queue int. & low & pending & DI or AI \\
	       &        &        &            &     & maskable & \\
\hline
   \end{tabular}
   \end{center}
\end{table}
%
%\end{document}

{\tt <}표~\ref{table:vec-type}{\tt >}은 각 형태의 벡터의 특성을 기술하고 있다.
Type 필드의 값이 클수록 처리의 우선 순위가 높다.
%
\subsection{E(Emergency)-Type}
긴급한 상황이 발생했을 경우 사용하는 인터럽트이다.
지정 인터럽트에서만 사용할 수 있으므로 8 비트의 벡터를 갖는다.
마스크시킬 수 없으며, 해당 프로세서로 가장 높은 순위의 인터럽트로 전달해야 한다.
%
\subsection{I(Immediate)-Type}
이 형태의 벡터는 인터럽트라기 보다는 인터럽트 버스를 통하여 
단순한 기능을 구현하기 위한 목적으로 사용한다.
지정 인터럽트에서만 사용할 수 있으므로 8 비트의 벡터를 갖는다.
대부분 처리기가 독자적으로 수행할 수 있는 단순한 기능을 구현하기 때문에 
해당 프로세서로 인터럽트를 전달하지 않을 수 있다.
마스크시킬 수 있으며, 마스크되어 있는 상태에서 인터럽트가 전송될 경우 그 인터럽트는 무시된다.
시스템의 시험이나 고장 진단의 목적으로 이용될 수 있다.
%
\subsection{N(Notify)-Type}\label{section:n-type}
Q-Type과 함께 다중 처리 운영 체제가 동작 중에 주로 사용될 수 있는 벡터들이다.
지정 인터럽트와 중재 인터럽트를 통하여 사용할 수 있다.  따라서 16 비트의 벡터 크기를 갖는다.
지정 인터럽트와 중재 인터럽트가 서로 벡터의 길이가 다르기 때문에
지정 인터럽트로 이 벡터를 전송할 경우는 낮은 자리의 한 바이트가 생략된 것으로 처리된다.
이 형태 특징은 우선 순위에 있어서 Q-Type보다 높고
인터럽트가 마스크될 경우 접수되지 않고 (No Pending) 요청기로 접수되지 않음을 알린다.
처리가 지연된 상태에서 새로운 인터럽트가 전송되면 이전에 도착한 인터럽트는 무시되며,  
이 경우 에러로 처리되지 않는다.
즉, 인터럽트가 처리되지 않은 상태에서 다시 이 형태의 인터럽트가 
전달되면 그전에 전달되었던 인터럽트는 무시되고 새로운 인터럽트를
처리하게 된다.
%
\subsection{Q(Queue)-Type}
N-Type보다 낮은 우선 순위를 갖으며 한번 전송된 인터럽트는
반드시 처리되어야 한다. 따라서 인터럽트가 마스크될 경우는
처리가 지연(Pending)된다. 지연된 상태에서 새로운 인터럽트가
전달되면 처리기의 저장 능력에 따라 Queue에 쌓아두거나,
Queue가 부족할 경우는 요청기로 전달된 인터럽트를 받을 수 없음을 알린다
N-Type과 같이 운영 체제에서 16 비트의 벡터를 정의하여 사용할 수 있다.
지정 인터럽트로 전달할 경우는 낮은 자리의 8 비트가 생략된 것으로 가정한다.
%
%
