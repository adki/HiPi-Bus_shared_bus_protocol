%%%%%
%%Title: HiPi+Bus V0.2 Chapter 4 Section 8
%%Creator: Ando Ki
%%CreationDate: April 1992
%%FileName: sec9
%%RelatedFile: ch4
%%%%%
\section{예외 처리 (Exception Handling)}
본 절에서는 인터럽트 버스 상에서 동작 중에 앞에서 정의되지 않은 상태가 발생했을 때
처리하는 방법에 대해 기술한다.

\subsection{전송 에러 (Transmission Error)}
인터럽트 전송 주기 안에서 처리기가 패리티 에러를 검출한 경우를 말한다.
처리기는 응답 단계를 통하여 요청기로 전송 중에 에러가 발생했음을 통고하고,
받은 인터럽트는 등록하지 않는다. 요청기는 전송 에러를 받으면
중재 주기 부터 다시 인터럽트 전송을 재시도 (Retry)한다.
재시도한 전송 주기에서 다시 동일한 에러를 검출할 경우는 고장이 발생했음을
시스템 콘트롤러에 전달하고 그 이후의 동작은 버스 규격에서 제시하지 않는다.
