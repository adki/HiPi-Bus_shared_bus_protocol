%%%%%
%%Title: HiPi+Bus V0.2 Chapter 4 Section 3
%%Creator: Ando Ki
%%CreationDate: April 1992
%%FileName: sec3
%%RelatedFile: ch4
%%%%%
\section{인터럽트의 종류}
인터럽트의 종류(interrupt class)는 인터럽트의 전송 방법에 따라 구분된다.
인터럽트의 종류는
인터럽트 처리기를 요청기가 선정하여 인터럽트를 전송하는 지정 인터럽트(direct interrupt)와
인터럽트 처리기를 요청기들 사이의 중재를 통하여 선정되도록 하는 
중재 인터럽트(arbitration interrupt)로 크게 분류할 수 있다.
중재 인터럽트는 전송 정보의 양에 따라 중재 인터럽트 1, 중재 인터럽트 0로 구분된다.
3 종류의 인터럽트는 인터럽트 전송 순서에 있어서 우선 순위가 있으며,
지정 인터럽트가 높은 우선순위를 가지며, 중재 인터럽트 1, 그리고 중재 인터럽트 0의 순서로 우선순위가 낮아진다.
인터럽트 종류는 인터럽트 버스 중재(interrupt bus arbitration)와
인터럽트 전송의 첫번재 단계(phase 0)에 Class 필드에 의해 정의된다.
Class 필드의 값은 {\tt <}표~\ref{table:int-class}{\tt >}와 같다.
%\documentstyle[a4]{hbook}
%\begin{document}
%
\begin{table}[htbp]
\caption{인터럽트 종류들}\label{table:int-class}
   \begin{center}
   \begin{tabular}{|l|l|r|l|} \hline
	interrupt class & interrupt name & num. of phases & priority\\
\hline \hline
	3 & direct interrupt & 5 & high \\
	2 & arbitration interrupt 1 & 22 & \\
	1 & arbitration interrupt 0 & {\it reserved\/} & low \\
	0 & no interrupt & 0 & \\
\hline
   \end{tabular}
   \end{center}
\end{table}
%
%\end{document}

%
\subsection{지정 인터럽트}
지정 인터럽트란 인터럽트 요청기가 인터럽트를 받을 처리기의 주소를 지정하여 
인터럽트를 전송하는 것을 말한다.
일반적으로 지정 인터럽트는 시스템의 제어(시험, 고장 진단 등)와 
그리고 실시간 응용등에 사용될 수 있다.
전송 순서에 있어서 가장 높은 우선 순위를 갖는다. 
4 가지 벡터 형태(Q-, N-, I-, E-type)를 모두 전송할 수 있다. \\
처리기를 지정하는데 있어서 두가지의 방법이 있는데, 하나의 인터럽트 처리기만을
지정하는 경우와 동시에 시스템 전체에 있는 모든 처리기를 지정하는 경우(broadcast)가 있다.
전자의 경우는 각각 처리기에 한정된 정보의 전달을 위해 사용될 수 있고,
후자는 시스템 전체에 위급한 상황등의 정보 전달을 위해 사용할 수 있다.
지정 인터럽트의 경우는 여러개의 처리기에 인터럽트가 전달되어도 
중재 인터럽트와 같은 처리기 사이의 중재 활동이 필요없다. \\
지정 인터럽트는 5 개의 단계로 구성이 되고 각 단계는 한개의 인터럽트 버스 기본 주기가 소요된다.
%
\subsection{중재 인터럽트 1}
중재 인터럽트 1(arbitration interrupt 1)은 다중 프로세서 환경의
효과적인 인터럽트 분배를 구현하기 위해 필요한
인터럽트 전송 방법의 한 형태이다. 이 방법은 인터럽트의 동적인 분배를 실현할 수 있으며,
분배 과정에서 프로세서의 불필요한 Context Switching를 막을 수 있도록 구현되어 있다.
다중 프로세서 운영 체제 환경에서 입출력의 제어와 프로세서 사이의 대화 수단 등으로 사용할 수 있다. \\
중재 인터럽트 1은 지정 인터럽트와는 달리 처리기를 지정하지 않고(프로세서의 모임만 지정),
한개 이상의 처리기에 인터럽트를 동시에 전달한 후, 처리기들끼리 중재를 하여
인터럽트를 접수할 처리기를 하나만 선정하도록 한다.
중재 과정에서 처리기들은 현재 자신의 상태(상위 프로세서 등의 상태 포함)에 따라
우선 순위를 결정하게 되는, 그 순간에 인터럽트를 처리하기에 가장 적절한 것이
중재의 최종 승자가 되도록 구현되어 있다. \\
중재 인터럽터 1은 22 개의 단계로 구성되어 있고 각 단계는 한개의 인터럽트 버스 기본 주기를 사용한다.
%
\subsection{중재 인터럽트 0}
중재 인터럽트 0(arbitration interrupt 0)은 인터럽트 버스의 기능성 확장을 위하여 마련된 것으로
현재 규격은 유보한다.
%
