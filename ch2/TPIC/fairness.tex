%%%%%
%%Title: HiPi+Bus V0.2 Chapter 2
%%Creator: Ando Ki
%%CreationDate: April 1992
%%FileName: ch2
%%RelatedFile:
%%%%%
%\documentstyle[11pt,doublespace,a4wide,psfig]{hbook}
%\setstretch{1.2}
%\begin{document}
%
\begin{figure}[htb]
\begin{center}
  \begin{picture}(240,257)(-120,-257)
	\thicklines
	\put(0,-25){\circle{20}}
	\put(-45,-100){\framebox(90,45){}}
		\put(0,-70){\makebox(0,0){구동되어 있는}}
		\put(0,-82){\makebox(0,0){중재요청신호의}}
		\put(0,-94){\makebox(0,0){수를 계산한다}}
	\put(0,-120){\line(-5,-2){45}}
	\put(0,-120){\line(5,-2){45}}
	\put(0,-156){\line(-5,2){45}}
	\put(0,-156){\line(5,2){45}}
		\put(0,-138){\makebox(0,0){0 또는 1 ?}}
	\put(-91,-202){\framebox(72,36){}}
		\put(-55,-181){\makebox(0,0){다음 중재 싸이클}}
		\put(-55,-193){\makebox(0,0){까지 기다린다}}
	\put(19,-202){\framebox(72,36){}}
		\put(55,-181){\makebox(0,0){중재를}}
		\put(55,-193){\makebox(0,0){수행한다}}
	\put(55,-232){\circle{20}}
	\put(0,-35){\vector(0,-1){20}}
	\put(0,-100){\vector(0,-1){20}}
	\put(-45,-138){\line(-1,0){10}} \put(-55,-138){\vector(0,-1){28}}
		\put(-47,-136){\makebox(0,0)[rb]{아니오}}
	\put(45,-138){\line(1,0){10}} \put(55,-138){\vector(0,-1){28}}
		\put(47,-136){\makebox(0,0)[lb]{예}}
	\put(55,-202){\vector(0,-1){20}}
	\put(-45,-202){\line(0,-1){10}} \put(-45,-212){\line(-1,0){56}}
		\put(-101,-212){\line(0,1){167}} \put(-101,-45){\vector(1,0){101}}
	\put(-120,-257){\framebox(240,257){}} % 외곽 box
  \end{picture}
\end{center}
  \caption{공정성 중재 규칙}\label{figure:fairness}
\end{figure}
%%%%
%\end{document}
%%%%
