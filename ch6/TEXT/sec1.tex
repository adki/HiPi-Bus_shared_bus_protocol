%%%%%
%%Title: HiPi+Bus V0.2 Chapter 6
%%Creator: Ando Ki
%%CreationDate: April 1992
%%FileName: sec1
%%RelatedFile: ch6
%%%%%
\section{개요}
시간규격에서는  백플레인에 구동되는 신호의 시간적 제약을 규정한다.
모든 시간규격은 백플레인에 나타나는 실제 신호를 기준으로 표시하며, 표시의 편의를 위해
토템폴 신호, 트라이스테이트 신호 그리고 오픈콜렉터 신호를 따로 구분하지 않고,
여러 신호가 모여서 동작하는 신호와 단일 신호를 따로 구분하지 않는다.
각 시간규격은 버스의 부하에 관계없이 최소값과 최대값의 범위를 넘지 않아야한다.

전달지연시간(propagation time)은 구동소자의 출력단 신호가
변하기 시작하여 백플레인에 나타나는 신호가 백플레인의 모든 곳에서
전기적규격에서 규정하는 $V_{OL}$ 이하로 안정될 때까지의 시간으로 규정한다.\footnote{BTL의 경우
90\%에서 10\%로의 하강시간이 최대 10$n$sec이고, 백플레인의 슬롯이 0.8inch 간격으로
21개 있을 때 5$n$sec/ft 이내의 전달 지연을 보장하면 신호가 백플레인을 최장으로 전달된다해도
10$n$sec 이내의 지연이 보장되므로 HiPi+Bus에서는 버스 전달 지연(bus propagation
time)을 최대 20$n$sec로 규정한다.}
%
