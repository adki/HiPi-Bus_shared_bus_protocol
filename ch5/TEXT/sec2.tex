%%%%%
%%Title: HiPi+Bus V0.2 Chapter 5
%%Creator: Ando Ki
%%CreationDate: April 1992
%%FileName: sec2
%%RelatedFile: ch5
%%%%%
\section{유틸리티 버스의 신호선}
%
%\documentstyle[a4]{hbook}
%\begin{document}
%
\begin{table}[htbp]
\caption{유틸리티 버스의 신호들}\label{table:ub-signal}
   \begin{center}
   \begin{tabular}{|l|r|l|} \hline
	Mnemonic & Size & Name \\
\hline \hline
	BCLK*		& 1 & Bus Clock \\
	GA$<$4..0$>$*   & 5 & Geographical Slot Address \\
	SFAIL*          & 1 & System Fail \\
	RST*            & 1 & Reset \\
	Tx$<$4..0$>$	& 5 & JTAG Boundary Scan Option \\ \hline
   \end{tabular}
   \end{center}
\end{table}
%
%\end{document}

%
\subsection*{Bus Clock : BCLK*}
버스 동작의 기준 시간이 되는 신호이다. 백플레인에서 공급된다.
각 슬롯에 도착되는 신호의 시간 차이 발생을 최소화하기 위하여
두개의 슬롯씩 별도의 신호를 두어 클럭을 공급한다. 단 슬롯 10번은
슬롯 8, 슬롯  9와 같이 연결된다.
%
\subsection*{Geographical Slot Address : GA$<$4..0$>$*}
슬롯 어드레스는 슬롯의 위치를 나타내는 5 비트의 신호들로써 슬롯에 삽입된 보드는
이 신호를 통하여 자신의 위치를 알 수 있게 된다.
슬롯의 어드레스는 주서브랙(main sub-rack)의 전면에서 볼 때 가장 왼쪽이 슬롯 0가 되고, 
순서적으로 할당되어 가장 오른쪽이 슬롯 20이 된다.
각 슬롯마다 독립적으로 자신의 위치를 나타내는 값을 버스 콘넥터에 제공해야 하므로
다른 버스의 신호와 달리 모든 슬롯을 연결시키는 신호가 아니다.
$<$표~\ref{table:ga}$>$은 보드가 꽂히는 쪽(백플레인의 전면)에서 바라본 슬롯의 위치와
슬롯 어드레스 값의 대응관계를 보여 주고 있다.
%
%\documentstyle[a4]{hbook}
%\begin{document}
%
\begin{table}[htbp]
\caption{슬롯 어드레스와 슬롯 위치}\label{table:ga}
   \begin{center}
   \begin{tabular}{|l|l l l l l l l l l c l l l l l l l l|} \hline
	Slot No. & 0&1&2&3&4&5&6&7&8&...&13&14&15&16&17&18&19&20 \\
\hline \hline
	GA{\tt <}4{\tt >}* &0&0&0&0&0&0&0&0&0&...&0&0&0&1&1&1&1&1 \\
	GA{\tt <}3{\tt >}* &0&0&0&0&0&0&0&0&1&...&1&1&1&0&0&0&0&0 \\
	GA{\tt <}2{\tt >}* &0&0&0&0&1&1&1&1&0&...&1&1&1&0&0&0&0&1 \\
	GA{\tt <}1{\tt >}* &0&0&1&1&0&0&1&1&0&...&0&1&1&0&0&1&1&0 \\
	GA{\tt <}0{\tt >}* &0&1&0&1&0&1&0&1&0&...&1&0&1&0&1&0&1&0 \\
\hline
   \end{tabular}
   \end{center}
\end{table}
%
%\end{document}

%
각 슬롯에 꽂히는 보드들은 초기화 과정에서 이 슬롯 어드레스를 읽어서 중재 번호, 인터럽트 버스의
인터럽트 요청기와 처리기의 주소, 그리고 기타 자신을 나타내는 번호(ID)로 사용한다.
%
\subsection*{System Reset : RST*}
시스템 전원을 올린 후 시스템 내의 모든 하드웨어의 초기화를 위하여 사용하며, 또한 시스템의 운영중
복구가 불가능한 에러가 발생했을 경우에도 하드웨어의 초기화를 위하여 사용한다.
시스템 콘트롤러에 의해서 구동되고 다른 모든 슬롯에 위치하는 모듈은 입력 신호로 사용된다.
이 신호가 구동될 경우 각 모듈은 수행 중인 작업을 중단하고 자신이 갖고 있는 모든 자원을 
초기화하여야 한다. 이 신호의 시간 규격은 최소한 100 {\it m\/}sec
동안 안정한 값을 구동하여야 한다는 것이다.
%
\subsection*{System Fail : SFAIL*}
시스템의 고장 신호는 국부적으로 복구가 불가능한 시스템의 고장이 발생했을 때, 시스템 콘트롤러에 고장의 발생을
알리는 신호로 사용된다. 보통 복구가 불가능한 에러는 동시에 여러 슬롯에서 발생할 수 있기 때문에
두개 이상의 모듈이 동시에 구동이 가능하다.
%
\subsection*{Boundary Scan Option : Tx$<$4..0$>$}
JTAG (Joint Test Action Group) 경계주사(boundary scan) 기능을
위한 핀이다. 정확한 기능과 규격은 IEEE Std.1149.1을 따르며 여기에서는
$<$표~\ref{table:jtag}$>$과 같이 사용핀을 정의한다.
%\documentstyle[a4]{hbook}
%\begin{document}
%
\begin{table}[htbp]
\caption{경계주사}\label{table:jtag}
   \begin{center}
   \begin{tabular}{|l|l|l|} \hline
	& define & description \\
\hline \hline
	Tx{\tt <}4{\tt >} & TRST* & Test Reset Input \\
	Tx{\tt <}3{\tt >} & TDO   & Test Data Out \\
	Tx{\tt <}2{\tt >} & TDI   & Test Data Input \\
	Tx{\tt <}1{\tt >} & TMS   & Test Mode Select Input \\
	Tx{\tt <}0{\tt >} & TCK   & Test Clock Input \\
\hline
   \end{tabular}
   \end{center}
\end{table}
%
%\end{document}

%
%
