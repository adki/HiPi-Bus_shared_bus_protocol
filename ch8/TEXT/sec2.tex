%%%%%
%%Title: HiPi+Bus V0.2 Chapter 8
%%Creator: Ando Ki
%%CreationDate: April 1992
%%FileName: sec2
%%RelatedFile: ch8
%%%%%
\section{백플레인}
백플레인의 크기는
%TBD\footnote{추후정의}
525.78$m$m$\times$486.3$m$m$\times$4$m$m
 (width$\times$heigh$\times$thickness)이다.
백플레인에는 보드가 꽂힐 슬롯, 파워바가 연결되어 전원이 공급될 접점,
주서브랙과 입출력서브랙이 고정될 지지영역등이 있다.
그리고 버스클럭 발생회로가 있고, 터미에이션 저항이 설치될 영역이 있다.
백플레인에는 보드가 꽂힐 21개 슬롯(슬롯 0 - 20)이 있고,
각 슬롯에는 보드와의 연결을 위한 백플레인 콘넥터(간접형 숫 콘넥터)가 설치된다.
백플레인을 전면에서 볼 때 가장 왼쪽부터 슬롯 0번(J0), 그 다음이 슬롯 1번(J1)
순이며 가장 오른쪽이 슬롯 20번(J20)이다.
각 슬롯 간의 간격은 0.8 inch 이며, 따라서 슬롯 0번에서 슬롯 20번까지의 거리는
16 inch(406.4 $m$m)이다.
백플레인은 슬롯에 꽂힌 보드들 사이의 신호 전송의 통로를 제공하고
또한 슬롯을 통하여 각 보드에 전원을 공급한다. 각 보드에 공급될 +5V 전원은
파워 바(Power Bar)를 통하여 전원 공급 장치로부터 공급받으며 이 파워 바와의
연결을 위해 백플레인의 위 아래에 각각 20개씩의 접점(+5V용 20개, GND용 20개)이 있다.
또한 +5V STB 전원용 파워 탭(Power Tab) 설치를 위한 접점이 있다.
백플레인은 주서브랙의 뒷면에 설치되며, 그 상세한 규격은 도면으로 첨부한다.
%
%\documentstyle[a4]{hbook}
%\begin{document}
%
\begin{table}[htbp]
\caption{백플레인의 기계적 규격}\label{table:backplane-spec}
   \begin{center}
   \begin{tabular}{|l l|} \hline
	\multicolumn{2}{|c|}{외관상 규격} \\ \hline
	Size & 525.78$m$m$\times$486.3$m$m$\times$4$m$m \\
	Number of Slot & 21 (J0 - J20) \\
	Interval between neighbor slots & 0.8 inch \\
	Interval from J0 to J20 & 16 inch \\ \hline
	\multicolumn{2}{|c|}{Degree of PCB layer} \\ \hline
	Signal & 4 \\
	Clock & 1 \\
	+5 & 2 \\
	SGND & 1 \\
	PGND & 2 \\
	Total & 10 \\ \hline
	\multicolumn{2}{|c|}{전원 공급용 접점수} \\ \hline
	+5V & 20 \\
	GND & 20 \\
	+5V Standby & 1 \\ \hline
   \end{tabular}
   \end{center}
\end{table}
%
%\end{document}
%%%%

%
%%%%%
