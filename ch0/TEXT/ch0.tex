%%%%%
%%Title: HiPi+Bus V1.1 Chapter 0
%%Creator: Ando Ki
%%CreationDate:
%%FileName: ch0
%%RelatedFile:
%%%%%
\documentclass[12pt]{book}
\usepackage{hangul}
\usepackage{a4wide}
\usepackage{psfig}
\usepackage{fancyheadings}
%%%
\pagestyle{fancy}
\setlength{\headrulewidth}{0pt}
\lhead{\leftmark}
\chead{}
\rhead{\rightmark}
\lfoot{무단복제금지 \psfig{figure=smalletri.ps} ETRI Proprietary}
\cfoot{}
\rfoot{\thepage}
\setcounter{secnumdepth}{4}
\setcounter{tocdepth}{4}
%
%\nofiles
%
\begin{document}
\def\HB{HiPi+Bus}	% define system bus name
%
\pagenumbering{roman}
\tableofcontents
\addcontentsline{toc}{chapter}{목차}
\newpage
\listoffigures
\addcontentsline{toc}{chapter}{그림목차}
\newpage
\listoftables
\addcontentsline{toc}{chapter}{표목차}
%
\newpage
\chapter*{변경이력}
\addcontentsline{toc}{chapter}{변경이력}
%
\begin{enumerate}
  \item 1993년 2월 12일 : Ver.1.0 작성완료
  \item 1993년 2월 22일 : BLK103-21-1.0 문서등록
  \item 1993년 8월 16일 : Ver.1.1 작성완료
	\begin{itemize}
	  \item 오자교정
	  \item BCLK$<$9..0$>$*를 BCLK*로 변경
	  \item DI$<$7..0$>$*의 의미를 확장함 (표3.10 참고)
	  \item WRB 전송형태의 의미를 확장함 (제3.2.1장 Transfer Type 설명 참고)
	  \item 제7장 전기적 규격에 TTL 관련 규격을 추가함
	  \item 제8장 기계적 규격의 내용을 변경함
	  \item 부록을 추가함
	\end{itemize}
  \item 1993년 8월 23일 : BLK103-21-1.1 문서등록
  \item 1993년 12월 16일 : Ver.2.0 작성완료
	\begin{itemize}
	  \item 오자교정
	  \item 제3장 데이터전송버스의 내용중 BE$<$15..0$>$* 부분 변경
	  \item 제8장 기계적 규격의 내용중 백플레인 크기와 주보드 크기를 정정함
	\end{itemize}
\end{enumerate}
%%%%%
\end{document}
%%%%%
