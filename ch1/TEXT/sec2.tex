%%%%%
%%Title: HiPi+Bus V0.2 Chapter 1 Section 2
%%Creator: Ando Ki
%%CreationDate: April 1992
%%FileName: sec2
%%RelatedFile: ch1
%%%%%
\section{블록 설계 개념}
\subsection{다중 프로세서를 지원하기 위한 고려사항}
\begin{itemize}
	\item Pended Protocols \\
		Pended 프로토콜은 하나의 데이터 전송을
		요청 단계(request phase)와 응답 단계(respond phase)로
		나눔으로써 메모리 접근 시간이 버스의 전송 속도(bus bandwidth)에
		영향을 주지 않도록 하는 방법이다. 
		메모리를 접근하는 동안 하나의 프로세서가 계속해서 버스를 점유하는
		것이 아니므로, 
		다수의 데이터 전송 요청이 동시에 발생할 수 있는 
		다중 프로세서 환경에서는 이 프로토콜이 절대적으로 유리하다.
	\item Synchronous - 16.5 MHz \\
		Pended 프로토콜은 데이터 전송을 요청과 응답의 두 단계로
		나누어 pipeline시킨 것이다.
		따라서 이와 같은 프로토콜을 구현하기에는 비동기형 제어 방식보다는 
		동기형 제어방식이 유리하다.
		동기형 제어 방식은 비동기형에 방식에 비해 기술의 발전에 따른 
		탄력적 성능 향상을 얻기 어렵지만, 신뢰도가 높고
		현재의 기술로 비동기형보다 나은 성능을 발휘할 수 있도록
		구현할 수 있다는 장점을 가지고 있다.
		클럭 주기는 단위 버스 동작이 수행되는 시간을 말하는
		것으로 60.6 {\it n\/}sec이다.
	\item Fairness arbitration \\
		다중 프로세서 시스템의 가장 큰 장점 중의 하나는 여러개의
		프로세서가 일을 분담하여 수행함으로써 얻어지는 
		빠른 응답이라 할 수 있다. 이와 같은 장점을 충분히 발휘하도록
		하기 위해서는 모든 프로세서가 동등한 자격을 갖고
		매 순간마다 일의 균형있는 분배 (dynamic load balancing)가
		이루어져야 한다.
		균형 분배는 운영 체제에서 많은 책임을 갖고 있지만,
		우선 하드웨어 수준에서의 동등한 자원 공유가 보장되어야 가능하다.
		버스는 가장 핵심이 되는 자원이므로 버스 사용의 공정한 중재가
		균형 분배의 시발점이라 할 수 있다.
		\HB에는 모든 버스 모듈이 공정하게
		버스를 사용할 수 있도록 중재 규칙을 정의하였다.
	\item Cache Coherency Protocols \\
		버스를 사용하는 모든 다중 프로세서 시스템의 가장 큰 문제는
		버스의 포화에 의한 시스템 성능 저하이다.
		버스가 포화되면 프로세서 수가 증가하여도 시스템의 성능은
		향상되지 않고 오히려 감소되기 때문이다.
		이와 같은 문제를 해결하기 위한 방법으로 공유 메모리를 사용하는
		다중 프로세서 시스템에서는 각 프로세서가 캐쉬를 갖게
		하여 캐쉬에 원하는 데이터가 존재하지 않을 경우에 한해서
		버스를 사용하게 함으로써 버스의 포화를 방지한다.
		그러나 프로세서마다 독립적으로 캐쉬를 사용하게 되면 
		캐쉬상에 존재하는 데이타가 메모리나 다른 캐쉬의 데이타와
		일관성이 깨어지는 문제(multi-cache consistency problem)가
		발생할 수 있으므로 이 문제를 해결하여야 한다.
		\HB에서는 write-back
		캐쉬의 동일성 유지 프로토콜을 지원한다.
	\item Interlock Protocols \\
		다중처리에서 필수적인 동기화를 지원하기 위한 잠금 프로토콜을 지원한다.
	\item Interrupt Broadcast and Arbitration \\
		공유 메모리를 사용하는 다중 처리 시스템 환경에 있어서의
		모든 프로세서들은 동등한 자격을 갖고 
		동작한다. 따라서 인터럽트의 요구와 처리에 있어서 단일
		프로세서 시스템과는 다른 기능들이 필요하게 된다.
		즉, 여러 프로세서가 모두 인터럽트를 처리할 수 있기 때문에
		여러 프로세서로 인터럽트를 보내는 방법(broadcast)과
		실질적으로 인터럽트를 처리할 프로세서를 선정하는
		기능(arbitration)이 필요하다.
		\HB는 이와 같은 인터럽트의 기능들을 제공하고 있으며,
		인터럽트를 모든 프로세서들 사이의
		비동기적인 통신 수단으로 이용할 수 있도록 일반화시켰다.
\end{itemize}
\subsection{고성능 버스를 위한 고려사항}
\begin{itemize}
	\item Bus bandwidth - 264 Mbytes/sec \\
		\HB는 264 Mbytes/sec의 지속적인 전송 용량(sustained
		bus bandwidth)을 갖는다.
	\item 128 bit data + 32 bit address, NonMultiplexed \\
		\HB는 128 비트의 데이터 전송을 위한 신호와
		32 비트 어드레스 전송을 위한 신호를 
		별도로 갖고 있다. 따라서 어드레스 전송을 위해 데이터의 전송선을
		사용하지 않기 때문에
		매 버스 클럭 주기(60.6 {\it n\/}sec)마다 128 비트의 데이터가
		전송될 수 있어서 264 Mbytes/sec의 
		데이터 전송 속도를 얻을 수 있다.
		32 비트의 어드레스 전송선을 갖기 때문에
		최대 4 Gbytes의 연속적인 어드레스 영역을 지원한다.
	\item Block Transfer \\
		캐쉬 라인 크기와 같은 크기인 64바이트 전송을 지원한다.
	\item All bytes are parity protected \\
		시스템의 신뢰도 향상을 위하여 버스에서는 모든 신호에
		에러 방어를 위한 신호를 추가한다.
		에러 방어의 대상은 순간적인 에러(transient or soft error)이며
		재시도(retry)에 의해 처리한다.
		고장으로 인한 계속적인 에러에 대해서는 처리 방법을 버스 규격에서
		별도로 정의하지 않는다.
		기본적으로 버스의 모든 신호는 바이트 단위의 패리티 비트를 갖는다.
		단, 신호의 특성상 바이트 단위로 동시에 구동되지 않는
		신호의 경우는 예외로 한다.
	\item 21 slots \\
		하나의 보드는 하나의 슬롯을 이용하여 버스상에 장착된다.
		따라서 슬롯의 수는 버스에 장착될 수 있는 최대 보드의 수를 말하고
		이것은 시스템의 구성을 제한한다.
		21 개의 슬롯은 13 개의 프로세서 보드 (데이터 처리 프로세서와 입출력 프로세서)와
		8 개의 메모리 보드가 장착되도록 한다.
	\item No mechanical switch and jumper \\
		이전에는 버스나 버스 모듈에
		다소의 유연성을 제공하기 위하여 기계적인 스윗치나 점퍼를 많이 사용되었다. 
		그러나 기계적 부품은 다른 부품에 비하여 신뢰도가 많이 떨어지기
		때문에 많은 시스템 고장의 원인이 되어왔다.
		이와 같은 이유로 \HB에서는 기계적인 부품을 없애고
		소프트웨어 제어가 가능한 레지스터를 사용하여 스윗치를 대신한다.
	\item BTL (Backplane Transceiver Logic) \\
		백플레인 구동에는 BTL 기술을 사용한다.
	\item I/O connector \\
		백플레인에 입출력 버스와의 연결을 용이하게 하기 위한 콘넥터를 둔다.
\end{itemize}
%%%%%
