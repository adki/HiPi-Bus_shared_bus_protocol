%%%%%
%%Title: HiPi+Bus V0.2 Chapter 1 Section 1
%%Creator: Ando Ki
%%CreationDate: April 1992
%%FileName: sec1
%%RelatedFile: ch1
%%%%%
\section{서론}
\subsection{개발 목적 및 내용}
공유메모리를 사용하는 다중프로세서 구조를 갖는 시스템에서 프로세서와 메모리 사이의
정보 통로로 사용할 공유버스를 설계하는 것이 본 문서의 목적이다.
\HB\footnote{Highly Pipelined Plus Bus : \HB :
/hai-pai-pl\'{$\land$}s-b$\land$s/}는 시스템 버스이며,
이러한 시스템 버스를 설계하는 목표는 다음과 같다.
\begin{itemize}
  \item 공유메모리 다중프로세서 구조를 지원 \\
	공유메모리를 사용하는 다중프로세서 구조에서 프로세서와 메모리를 버스를
	이용하여 연결시킬때 시스템 버스는 데이터 전송 프로토콜, 전송속도,
	중재방법, 그리고 인터럽트 등 버스의 각 부분에서 다중프로세서를 지원하기
	위한 고려가 필요하다.
  \item 고성능 컴퓨터 지원 \\
	목표로하는 다중프로세서의 성능이 고성능이기 위해서는 시스템 버스도 이에
	어울리는 성능과 기능을 제공해야 하며 이를 위해 데이터 전송 능력,
	데이터 전송의 신뢰도, 시스템 구성 능력, 시스템 유지 보수를 위한 기능 등이
	고려되어야 한다.
\end{itemize}
\subsection{본 문서의 구성}
본 문서의 구성은 제 1 장에서 전체적인 구성과 설계 개념을 기술하고,
제 2 장에는 중재버스, 제 3 장에는 데이터 전송 버스, 제 4 장에는
인터럽트 버스, 제 5 장에는 유틸리티 버스, 제 6 장에는 시간 규격,
제 7 장에는 전기적 규격, 제 8 장에는 기계적 규격을 담고 있다.
\subsection{관련문서}
\begin{description}
  \item[(1)] 고속중형컴퓨터 (주전산기 III) 요구사항 정의서, SYS000-20-3.0,
	컴퓨터시스템연구실, 시스템공학연구부, 한국전자통신연구소, 1993.
  \item[(2)] 고속중형컴퓨터 (주전산기 III) 시스템 설계서, SYS000-21-3.0,
	컴퓨터시스템연구실, 시스템공학연구부, 한국전자통신연구소, 1993.
  \item[(3)] 고속중형컴퓨터 (주전산기 III) 하드웨어 서브시스템 설계서, SUS100-21-2.0,
	컴퓨터시스템연구실, 시스템공학연구부, 한국전자통신연구소, 1994.1.
\end{description}
\subsection{문서트리}
%
\begin{figure}[htb]
%\begin{center}
  \begin{picture}(400,170)(-200,-170)
	\thicklines
	\put(-50,-30){\framebox(100,20){요구사항 정의서}}
	\put(-50,-70){\framebox(100,20){시스템 설계서}}
	\put(-194,-110){\framebox(144,20){하드웨어 서브시스템 설계서}}
	\put(50,-110){\framebox(144,20){소프트웨어 서브시스템 설계서}}
	\put(-196,-171){\framebox(72,32){}} %외곽박스
	\put(-195,-170){\framebox(70,30){}}
		\put(-160,-153.5){\makebox(0,0)[b]{시스템버스}}
		\put(-160,-156.5){\makebox(0,0)[t]{블록설계서}}
	\put(-115,-170){\framebox(70,30){}}
		\put(-80,-153.5){\makebox(0,0)[b]{주처리장치}}
		\put(-80,-156.5){\makebox(0,0)[t]{블록설계서}}
	\put(-35,-170){\framebox(70,30){}}
		\put(0,-153.5){\makebox(0,0)[b]{주기억장치}}
		\put(0,-156.5){\makebox(0,0)[t]{블록설계서}}
	\put(45,-170){\framebox(70,30){}}
		\put(80,-153.5){\makebox(0,0)[b]{시스템제어기}}
		\put(80,-156.5){\makebox(0,0)[t]{블록설계서}}
	\put(125,-170){\framebox(70,30){}}
		\put(160,-153.5){\makebox(0,0)[b]{입출력처리기}}
		\put(160,-156.5){\makebox(0,0)[t]{블록설계서}}
	\put(0,-30){\line(0,-1){20}}
	\put(0,-70){\line(0,-1){10}}
	\put(-122,-80){\line(1,0){244}}
	  \put(-122,-80){\line(0,-1){10}}
	  \put(122,-80){\line(0,-1){10}}
	\put(-160,-125){\line(1,0){320}}
	  \put(-122,-110){\line(0,-1){15}}
	  \put(-160,-125){\line(0,-1){15}}
	  \put(-80,-125){\line(0,-1){15}}
	  \put(0,-125){\line(0,-1){15}}
	  \put(80,-125){\line(0,-1){15}}
	  \put(160,-125){\line(0,-1){15}}
  \end{picture}
%\end{center}
  \caption{문서트리}\label{figure:doc-tree}
\end{figure}
%

\subsection{참고자료}
\begin{description}
  \item[(1)] 컴퓨터구조연구실, {\em TICOM 시스템 버스 사용자 지침서 (TD88-6121-133.B)},
	시스템구조연구부, 한국전자통신연구소, 1990.11.
  \item[(2)] 컴퓨터구조연구실, {\em 시스템 버스 표준 인터페이스 설계서 Ver.2.0 (TM93-CA-017.B)},
	시스템구조연구부, 한국전자통신연구소, 1993.7.
  \item[(3)] 컴퓨터구조연구실, {\em TTL 회로 설계의 고려사항 (TM92-CA-039)},
	시스템구조연구부, 한국전자통신연구소, 1992.6.26.
  \item[(4)] 컴퓨터구조연구실, {\em HiPi+Bus의 임계시간 (TM92-CA-053)},
	시스템구조연구부, 한국전자통신연구소, 1992.11.16.
  \item[(5)] 컴퓨터구조연구실, {\em BTL 기초조사 (TM93-CA-001.B)},
	시스템구조연구부, 한국전자통신연구소, 1993.2.
  \item[(6)] IEEE, {\em IEEE Standard Test Access Port
	and Boundary-Scan Architecture}, IEEE std.1149.1, 1990.
  \item[(7)] IEEE, {\em IEEE Standard for Electrical Characteristics of
	Backplane Transceiver Logic (BTL) Interface Circuits},
	IEEE std.1194.1, 1991.
  \item[(8)] 컴퓨터구조연구실, {\em Semaphore Cache {\it (Draft)\/}},
	시스템구조연구부, 한국전자통신연구소, 1993.
  \item[(9)] 컴퓨터구조연구실, {\em 인터럽트 관련 모듈 설계서 {\it (Draft)\/}},
	시스템구조연구부, 한국전자통신연구소, 1993.
  \item[(10)] 컴퓨터구조연구실, {\em 고속중형컴퓨터 클럭 서브 시스템 (TM93-CA-020)},
	시스템구조연구부, 한국전자통신연구소, 1993.4.
\end{description}
%
%%%%%
