%%%%%
%%Title: HiPi+Bus V0.2 Chapter 3 Section 1
%%Creator: Ando Ki
%%CreationDate: April 1992
%%FileName: sec4
%%RelatedFile: ch3
%%%%%
%\documentstyle[doublespace,a4wide]{hbook}
%\setstretch{1.2}
%\pagestyle{headings}
%\begin{document}
%\pagenumbering{arabic}
%\setcounter{chapter}{1}
%
\section{예외 처리 (Exception Handling)}
예외 처리란 지금까지 정의한 동작 규정 중에서 정상적으로 동작이 완료되지 않고
에러가 발생했거나 특별한 상황이 발생했을 때, 이의 처리를 말한다. 

\subsection{패리티 에러 (Parity Error)}
패리티는 데이터 전송 버스의 신호 전송에 있어서 에러의 발생을 검출하기 위하여 사용한다.
패리티 에러의 검출은 기본 주기 단위로 이루어지고,
패리티 에러의 처리는 전송 형태 단위의 재시도 (Retry)를 원칙으로 한다.
즉 데이터 전송 중 패리티 에러는 기본 주기의 끝 단계에서 상태 버스를 통하여 보고되거나
자체적으로 검출되고, 이의 처리는 그 기본 주기를 포함하는 전송 형태 단위로 재시도한다.

데이터 전송 중의 에러의 발생은 전송 형태 별로 약간씩 처리 방법이 다를 수 있는데
에러 처리의 관점에서 볼 때 기본 주기로 나눌 수 있다.

\subsubsection{어드레스 기본 주기에서 패리티 에러가 검출될 경우}
RP는 상태 버스를 통하여 RQ로 에러의 검출을 알리고 (AACK$<$1..0$>$* = 0)
받은 어드레스 버스의 정보를 무시한다. 어드레스에 패리티 에러가 검출되어 어떤 RP가 
선택되었는지를 알 수 없을 경우는 응답하지 않는다. 즉 AACK{\tt <1}..0$>$*를 구동하지 않는다.
(응답 신호를 구동하지 않으면 버스 상의 신호는 negative logic이기 때문에
RQ는 AACK$<$1..0$>$* = 0 인 것으로 받게 된다.)
RQ는 에러의 검출을 받으면 동일한 어드레스 기본 주기를 다시 시도한다 (재시도).
만약 연속하여 두번 에러 응답을 받을 경우는 해당 프로세서로 데이터 전송 버스의 에러를 알린다.
그 후의 처리는 프로세서의 버스 에러 루틴의 기능에 따른다.

\subsubsection{데이터 기본 주기에서 패리티 에러가 검출될 경우}
RQ는 RP로 부터 받은 데이터에서 패리티를 확인하여 에러 여부를 알 수 있다.
패리티 에러가 발생하면, 받은 데이터를 무시하고 어드레스 주기 부터 다시 수행한다.
따라서 이 경우는 RP는 데이터 전송 중에 발생한 에러 처리에 대하여 관여되지 않는다.
만약 재시도한 주기에서 또다시 에러가 발생할 경우는 해당 프로세서로 데이터 전송 버스의 에러를 알린다.
그 후의 처리는 프로세서의 버스 에러 루틴의 기능에 따른다.

블록 전송의 경우 전송되는 특성 데이터에 오류가 발생해도 일단 데이터 기본 주기를 끝까지 계속하고
다시 어드레스 주기 부터 재시도한다.

\subsubsection{어드레스 데이터 주기에서 패리티 에러가 검출될 경우}
RQ는 RP로 부터 오는 상태 버스에 의해 어드레스 버스와 데이터 버스의 전송 정보에 대한
에러 보고를 받는다 (AACK$<$1..0$>$* = 0, DACK* = 0).
RP가 에러를 검출하면 오류를 RQ로 보내고 어드레스 데이터 주기를 끝까지 수행한다.
RQ는 에러를 보고 받으면, 수행 중인 어드레스 데이터 기본 주기의 수행을 끝까지 마치고
처음부터 재시도를 한다.
만약 재시도한 주기에서 또다시 에러가 발생할 경우는 해당 프로세서로 데이터 전송 버스의 에러를 알린다.
그 후의 처리는 프로세서의 버스 에러 루틴의 기능에 따른다.

\subsection{Timeout Error}
Timeout 에러는 어드레스 주기의 전송을 마치고 데이터를 기다리는 RQ가 일정한 시간이
지난 후에도 데이터가 오지 않을 경우 무한히 기다리는 것을 방지하기 위한 에러이다.
어드레스 버스의 정보를 받은 RP가 해당 어드레스의 데이터를 접근하다가 문제가 발생하여
데이터를 응답하지 않을 경우 발생된다.
RQ가 어드레스 주기를 수행시키고 난 후 기다리는 시간은 조정이 가능하도록 구현하며,
최소한 데이터 접근 시간, 데이터 버스 사용상 지연 시간 등을 더한 것 보다 커야한다.

에러의 처리는 어드레스 주기 부터 재시도를 하며, 똑같은 에러가 발생될 경우는 패리티 에러의 경우와 같다.

\subsection{Busy 응답}
Busy 응답이란 어드레스 전송 주기에서 전송된 어드레스 버스의 정보를 처리할 수 없을 경우
발생한다 (AACK$<$1..0$>$* = 1).
예를 들면, 메모리 제어기가 이전에 전송되어 온
요청을 처리하기 위하여 새로 도착한 요청을 처리할 수 없을 경우를 말한다.
따라서 Busy 응답은 에러의 발생을 말하는 것이 아니며, 어드레스 전송은 제대로 되었으나
RP의 상태가 지금 당장 이를 처리하지 못 한다는 것을 의미한다.

RQ는 이와같은 응답을 받았을 경우 중재 주기에서 공정성 규칙을 무시하고 중재 부터 재시도한다.
그러나 경우에 따라 오랜 동안 Busy 응답만을 받고 메모리 보드 참조에 실패할 경우
ABINH* 신호를 이용할 수 있다. 그래도
계속해서 같은 응답을 받을 경우는 RP에 이상이 발생한 것으로 간주하여 프로세서로 RP의 에러를 알린다.
그 후의 처리는 프로세서의 RP 에러 루틴의 기능에 따른다.

\subsection{기타 에러}
기타 에러는 전송 규격에 정의되지 않은 신호의 조합을 발견했을 때와 그밖의 에러를 말한다.
모든 에러는 일단 순간적인 외부의 잡음으로 간주하고 재시도를 한다.
연속한 에러가 검출될 경우는 프로세서로 에러의 보고를 한다.
%%%
